% To produce pdf under linux, run
% pdflatex HW2.tex 
  
\documentclass{article}
\usepackage{amsmath,amssymb}
\usepackage{graphicx}
\usepackage{url}
\usepackage{listings}
\usepackage{color}
\usepackage{upquote}
\usepackage{courier}
\usepackage{caption}
\usepackage{verbatim} 
\usepackage{hyperref}
\usepackage{qtree}
\newcommand\Colorhref[3][blue]{\href{#2}{\small\color{#1}#3}}

\usepackage{array}
\setlength\arrayrulewidth{.4pt}
\setlength\tabcolsep{3pt}
\newcolumntype{L}[1]{>{\centering\let\newline\\\arraybackslash\hspace{0pt}}m{#1}}
\newcolumntype{C}[1]{>{\centering\let\newline\\\arraybackslash\hspace{0pt}}m{#1}}
\newcolumntype{R}[1]{>{\centering\let\newline\\\arraybackslash\hspace{0pt}}m{#1}}
\newcolumntype{M}[1]{>{\centering\let\newline\\\arraybackslash\hspace{0pt}}m{#1}}

\definecolor{dkgreen}{rgb}{0,0.6,0}
\definecolor{gray}{rgb}{0.5,0.5,0.5}
\definecolor{mauve}{rgb}{0.58,0,0.82}

\lstset{frame=shadowbox,
	rulesepcolor=\color{black},
  	language=bash,
  	aboveskip=3mm,
  	belowskip=3mm,
  	showstringspaces=false,
  	columns=flexible,
  	basicstyle={\small\ttfamily},
  	numbers=none,
  	numberstyle=\tiny\color{gray},
  	keywordstyle=\color{blue},
  	commentstyle=\color{dkgreen},
  	stringstyle=\color{red},
  	breaklines=true,
  	breakatwhitespace=true
  	tabsize=3
}
\title{Algorithm Assignment 1}
\author{Seshagiri Prabhu}

\begin{document}
\maketitle
\begin{center}
\href{mailto:seshagiriprabhu@gmail.com}{seshagiriprabhu@gmail.com}
\end{center}
\date{}
\section{Time complexity of program}
1. Derive the time complexity of the algorithm using Asymptotic analysis.


\begin{lstlisting}
# /usr/bin/python 

def bst(self, search, choice):
	''' Performs binary search based on the choice '''
	
	current = self.root
	while True:
		if search < current.trainNumber:  # C * n / 2
    		if current.left == None:     
        		break
	        else:
    	        current = current.left
                   
        elif search > current.trainNumber:  # C * n / 2
           	if current.right == None:       
            	break
	        else:
    	        current = current.right
    	        
        elif search == current.trainNumber: # C * 1
            break
                
        else:	# C * 1
        	break
        	
\end{lstlisting}

Lets take a simple example in which we search for node-41 from a huge BST.
\Tree[.Root [.Node-1 [.Node-3 Node-7 ]
               [.Node-4 [.Node-8  ]]]
          [.Node-2 [.Node-5 Node-9 ]
                [.Node-6 [.Node-10 [.Node-17 Node-26 ]
                           [.Node-18 [.Node-28 Node-29 ]
                                [.Node-39 [.Node-40 Node-78 ]
                                      \qroof{\textit{search this}}.Node-41 ]]]]]]
                                      
The peculiarity of this searching technique is that at each levels half of the nodes are eliminated.\\

This is the recursive equation of the searching algorithm we have used.
$T(n)$ = $T(n/2)$ + 1 \\


According to 2nd master's theorem:\\
$f(n)$ = $\theta(n^{\log _2 \left(1 \right)})$ \\
$(f(n)$ = $\theta(1)$

Hence \textbf{$T(n)$ = $\theta(f(n))$ = $\theta(\log  \left(n \right))$}           

\section{Implementation of the program}
The implementation of my program is available in my \Colorhref{https://github.com/seshagiriprabhu/advanced-computer-algorithm}{github} (under search-tree directory) repository.

\section{Graph}
\begin{flushright}
	\begin{tabular}{| L{4cm} | C{4cm} | R{4cm} |}
		\hline \textbf{Input Size} & \textbf{Node insertion time} & \textbf{Binary search time}   \\
		\hline 1000000 & 33.216& 18.831 \\
		\hline 100000 & 2.554 & 0.572 \\
		\hline 10000 & 0.226 & 0.046 \\
		\hline 1000 & 0.026 & 0.004 \\
		\hline 100 & 0.002	& 0.001 \\
		\hline 10 & 0 & 0 \\
		\hline
	\end{tabular}
\end{flushright}
\includegraphics[scale=1]{chart.png} 
\end{document}
