% To produce pdf under linux, run
% pdflatex HW2.tex 
  
\documentclass{article}
\usepackage{amsmath,amssymb}
\usepackage{graphicx}
\usepackage{url}
\usepackage{listings}
\usepackage{color}
\usepackage{upquote}
\usepackage{courier}
\usepackage{caption}
\usepackage{verbatim} 
\usepackage{hyperref}
\usepackage{qtree}
\usepackage{sagetex}
\newcommand\Colorhref[3][blue]{\href{#2}{\small\color{#1}#3}}

\usepackage{array}
\setlength\arrayrulewidth{.4pt}
\setlength\tabcolsep{3pt}
\newcolumntype{L}[1]{>{\centering\let\newline\\\arraybackslash\hspace{0pt}}m{#1}}
\newcolumntype{C}[1]{>{\centering\let\newline\\\arraybackslash\hspace{0pt}}m{#1}}
\newcolumntype{R}[1]{>{\centering\let\newline\\\arraybackslash\hspace{0pt}}m{#1}}
\newcolumntype{M}[1]{>{\centering\let\newline\\\arraybackslash\hspace{0pt}}m{#1}}

\definecolor{dkgreen}{rgb}{0,0.6,0}
\definecolor{gray}{rgb}{0.5,0.5,0.5}
\definecolor{mauve}{rgb}{0.58,0,0.82}

\lstset{frame=shadowbox,
	rulesepcolor=\color{black},
  	language=bash,
  	aboveskip=3mm,
  	belowskip=3mm,
  	showstringspaces=false,
  	columns=flexible,
  	basicstyle={\small\ttfamily},
  	numbers=none,
  	numberstyle=\tiny\color{gray},
  	keywordstyle=\color{blue},
  	commentstyle=\color{dkgreen},
  	stringstyle=\color{red},
  	breaklines=true,
  	breakatwhitespace=true
  	tabsize=3
}
\title{Algorithm Assignment 3}
\author{Seshagiri Prabhu}

\begin{document}
\maketitle
\begin{center}
\href{mailto:seshagiriprabhu@gmail.com}{seshagiriprabhu@gmail.com}
\end{center}
\date{}
\section{Time complexity of program}
1. Derive the time complexity of the algorithm using Asymptotic analysis.


\begin{lstlisting}
# /usr/bin/python

#V = vertex list, A = adjacency list, r = root
def prim(V, A, r): # O(V ^ 3)
    ''' A function to calculate the MST using prim's algorithm '''
    u = 0
    v = 0
    result =[]
    
    # initialize and set each value of the array P (pi) to none
    # pi holds the parent of u, so P(v)=u means u is the parent of v
    P=[None]*len(V)

    # initialize and set each value of the array K (key) to some large number (simulate infinity)
    K = [999999]*len(V)

    # initialize the min queue and fill it with all vertices in V
    Q=[0]*len(V)
    for u in range(len(Q)):
        Q[u] = V[u]

    # set the key of the root to 0
    K[r] = 0
    decreaseKey(Q, K)    # maintain the min queue

    # loop while the min queue is not empty
    while len(Q) > 0:	# O(V ^ 2)  
    	# pop the first vertex off the min queue 
        u = extractMin(Q)    # O(Constant)
        result.append(u)
        # loop through the vertices adjacent to u
        Adj = adjacent(A, u)  # O(V)
        for v in Adj:         # O(V ^ 3)
            w = A[u][v]    # get the weight of the edge uv
            #print "w(" + chr(u+97) + ', ' + chr(v+97) + '): ' + str(w)

            # proceed if v is in Q and the weight of uv is less than v's key
            if Q.count(v) > 0 and w < K[v]:
                # set v's parent to u
                P[v] = u
                # v's key to the weight of uv
                K[v] = w
                # maintain the min queue
                decreaseKey(Q, K)    # O(V)
    return result

\end{lstlisting}

As the adjacency matrix is used, the complexity is in polynomial order:- $O(V^{2})$ \\

\begin{sagesilent}
H = [(100, 0.105358123800000),
 (200, 0.761367082600000),
 (300, 2.51947116850000),
 (400, 5.99299383160000),
 (500, 11.6706569195000),
 (600, 20.2638969421000),
 (700, 31.8211338520000),
 (800, 47.3926920891000),
 (900, 67.4489781857000),
 (1000, 92.3331990242000)]
\end{sagesilent}

\begin{flushleft}
\sageplot[scale=0.61]{list_plot(H, plotjoined=True, color="red", legend_label="Prim's using Adjacency matrix, Complexity: $O(V^{2})$ ", frame=True,axes_labels=['No of vertices','Time taken (s)'], axes=False)}
\end{flushleft}

%\sageplot[scale=0.61]{list_plot(H, plotjoined=True, color="red", legend_label="Prim's using Adjacency matrix, Complexity: $O(V^{2})$ ", frame=True,axes_labels=['No of vertices','Time taken (s)'], axes=False)}

\begin{flushleft}
\sageplot[scale=0.70]{plot(lambda x:x*log(x,e), 1, 1000, color="red", frame=True, axes_labels=['No of vertices','Time taken (ms)'], axes=False, legend_label="Prim's using Adjacency matrix, Complexity: $O(Elog(V))$ ")}
\end{flushleft}
%, color="red", legend_label="Prim's using binary heap, Complexity: $O(E\log(V))$", frame=True, axes_labels=['No of vertices','Time taken (ms)'], axes=False)}

\end{document}
